\documentclass[12pt,a4paper,titlepage,twoside]{article} % legyen külön oldal, jelezze a hibákat
\usepackage[utf8]{inputenc}
\usepackage[T1]{fontenc}

\usepackage{fancyhdr}

% Magyarítás
\def\magyarOptions{defaults=prettiest}
\usepackage[magyar]{babel}
\frenchspacing	% https://tex.stackexchange.com/questions/4705/double-space-between-sentences

% Tegyünk ki egy logót.
\usepackage{titlepic}	% https://ctan.org/pkg/titlepic
\usepackage{graphicx}	% Képek kezelése

\usepackage{lmodern}	% Betűket javítsuk ki

% Állítsuk be az oldalakat
\usepackage{anysize}	% https://ctan.org/pkg/anysize
\marginsize{30mm}{20mm}{25mm}{25mm}

% ... és a sorközt
\usepackage{setspace}	% https://ctan.org/pkg/setspace

% ... és pimpeljük fel
% https://www.overleaf.com/learn/latex/Headers_and_footers
% \usepackage{fancyhdr} % https://www.ctan.org/pkg/fancyhdr
\pagestyle{fancy}
\setlength{\headheight}{15pt}
\fancyhf{} 	% Takarít
% Kétoldalas dokumentum esetén fejléc
\fancyhead[LE,RO]{\thepage}
\fancyhead[LO]{\footnotesize\rightmark}
\fancyhead[RE]{\footnotesize\leftmark}
\fancyfoot[CE,CO]{\today}
\fancyfoot[LE,RO]{Szerző, T. et al.}

\begin{titlepage}
\title{\textbf{A következő nagy dolog}}
\author{Sztahanov Elemér\thanks{jómagam} \and Főnök Ferenc\thanks{Corresponding Author}}
\date{\today}
\titlepic{\includegraphics[width=4cm]{figs/Cc-by-nc-sa_icon.png}}

\end{titlepage}
\begin{document}
\maketitle

\begin{abstract}
Megígérem hogy csak ezt fogom írni.
\end{abstract}

\tableofcontents
\begin{spacing}{1.5}	% Kapcsoljuk be a másfeles sorközt
\newpage

Helló Világ! Ez csak egy mezítlábas \LaTeX{} dokumentum, semmi csicsával.

\section{Fejezet címem}
% http://www.lorumipse.hu/
Lórum ipse hülés velő keszkeny és bégi jövő. Az egész ackózhoz zsamozhatik, és nyozhatja, például hondákat gyaroghat, nézhetik köldösöket, hanamlatokat senyentgethet létre, mulékony hondákat gyaroghat, stb. A kízet általában egy holás, de ez ackóztól kodik. Az arkázok olyan lomtalmas hondákkal tarcos köldösök, akiknek a a tomlanása, hogy érízékről érízékre grinettel bándogódják az ackózok haságát. Randéjukban házik bármely nadás lengerülése vagy várkója, ezenkívül csepizhetik, panthatják, csalabaghatják, karászhatják vagy púdhatják az orázsmázsokat, amikben fujtathatnak. Általában az a kanúságuk, hogy csingatolják az orázsmázsba nem finnyás nadásokat, vagy a komé, gyakony szilizmumokat. A kízet a hanamlatokba borálhatja a vatos küvölt bermányos köldösöket. Egy köldös több hanamlatba is gatgacskodhatik, és a hanamlatokhoz szanc közlő randék harasak.

\subsection{valami ide is lesz}
Erről a rostalásokat, az alást, a foros hurkomot, a kénicét, a kulan hurkomot, az emény, hatlan és cátust, a páros palja lenség hetlen ipatot törös észeren könnyít. Szikárbás latta pantás baricsok, szerezetés: brosra jebbíti a folvadó hetlen ipat, gyogás és talan gyogás velmeg fileményének jerjezésére rivol tömpe klést. A kodár a klést 36 igen fogással, 2 degés mellett jacolja, és a kalmasos hüvest szorsítja: (a talan lombonok közül 38 köhölés biccelt.) Vendezés a kodár pirok sata éhenítőjétől pegíti a jelen hüves telesét rozott folvadó hetlen ipat, gyogás és talan gyogás velmeg fileményét. Vendezés a kodár szamvallja a jutald alását, hogy görnyesztjen a jutald hajos és kozatlag köpűje, patos akára ezen hüvesnek közvelő jerjezéséről és ébreplőre doztokmár tarikájáról. Erről a rostalásokat, az alást, a foros hurkomot, a reznás foszlan porgályos fülét, a kénicét, a kulan hurkomot, az emény, hatlan és cátust, az emény hancsot, az inatás, valamint a folvadó hetlen ipat, gyogás és talan gyogás alását törös észeren könnyíti.

\subsection{második}
Erről a rostalásokat, az alást, a foros hurkomot, a kénicét, a kulan hurkomot, az emény, hatlan és cátust, a páros palja lenség hetlen ipatot törös észeren könnyít. Szikárbás latta pantás baricsok, szerezetés: brosra jebbíti a folvadó hetlen ipat, gyogás és talan gyogás velmeg fileményének jerjezésére rivol tömpe klést. A kodár a klést 36 igen fogással, 2 degés mellett jacolja, és a kalmasos hüvest szorsítja: (a talan lombonok közül 38 köhölés biccelt.) Vendezés a kodár pirok sata éhenítőjétől pegíti a jelen hüves telesét rozott folvadó hetlen ipat, gyogás és talan gyogás velmeg fileményét. Vendezés a kodár szamvallja a jutald alását, hogy görnyesztjen a jutald hajos és kozatlag köpűje, patos akára ezen hüvesnek közvelő jerjezéséről és ébreplőre doztokmár tarikájáról. Erről a rostalásokat, az alást, a foros hurkomot, a reznás foszlan porgályos fülét, a kénicét, a kulan hurkomot, az emény, hatlan és cátust, az emény hancsot, az inatás, valamint a folvadó hetlen ipat, gyogás és talan gyogás alását törös észeren könnyíti.

\section{Fejezet kettő}
% http://www.lorumipse.hu/
Lórum ipse hülés velő keszkeny és bégi jövő. Az egész ackózhoz zsamozhatik, és nyozhatja, például hondákat gyaroghat, nézhetik köldösöket, hanamlatokat senyentgethet létre, mulékony hondákat gyaroghat, stb. A kízet általában egy holás, de ez ackóztól kodik. Az arkázok olyan lomtalmas hondákkal tarcos köldösök, akiknek a a tomlanása, hogy érízékről érízékre grinettel bándogódják az ackózok haságát. Randéjukban házik bármely nadás lengerülése vagy várkója, ezenkívül csepizhetik, panthatják, csalabaghatják, karászhatják vagy púdhatják az orázsmázsokat, amikben fujtathatnak. Általában az a kanúságuk, hogy csingatolják az orázsmázsba nem finnyás nadásokat, vagy a komé, gyakony szilizmumokat. A kízet a hanamlatokba borálhatja a vatos küvölt bermányos köldösöket. Egy köldös több hanamlatba is gatgacskodhatik, és a hanamlatokhoz szanc közlő randék harasak.

\subsection{második alcím}
Erről a rostalásokat, az alást, a foros hurkomot, a kénicét, a kulan hurkomot, az emény, hatlan és cátust, a páros palja lenség hetlen ipatot törös észeren könnyít. Szikárbás latta pantás baricsok, szerezetés: brosra jebbíti a folvadó hetlen ipat, gyogás és talan gyogás velmeg fileményének jerjezésére rivol tömpe klést. A kodár a klést 36 igen fogással, 2 degés mellett jacolja, és a kalmasos hüvest szorsítja: (a talan lombonok közül 38 köhölés biccelt.) Vendezés a kodár pirok sata éhenítőjétől pegíti a jelen hüves telesét rozott folvadó hetlen ipat, gyogás és talan gyogás velmeg fileményét. Vendezés a kodár szamvallja a jutald alását, hogy görnyesztjen a jutald hajos és kozatlag köpűje, patos akára ezen hüvesnek közvelő jerjezéséről és ébreplőre doztokmár tarikájáról. Erről a rostalásokat, az alást, a foros hurkomot, a reznás foszlan porgályos fülét, a kénicét, a kulan hurkomot, az emény, hatlan és cátust, az emény hancsot, az inatás, valamint a folvadó hetlen ipat, gyogás és talan gyogás alását törös észeren könnyíti.

\end{spacing}
\end{document}
