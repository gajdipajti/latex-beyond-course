\documentclass[12pt,a4paper,titlepage,draft]{article} % legyen külön oldal, jelezze a hibákat
% \usepackage[utf8]{inputenc}
% \usepackage[T1]{fontenc}

% \renewcommand{\abstractname}{Összefoglaló}

\title{\textbf{A következő nagy dolog}}
\author{Sztahanov Elemér\thanks{jómagam} \and Főnök Ferenc\thanks{Corresponding Author}}
\date{\today}

\begin{document}
\maketitle

\tableofcontents

\begin{abstract}
Megígérem hogy csak ezt fogom írni.
\end{abstract}

Helló Világ! Ez csak egy mezítlábas \LaTeX{} dokumentum, semmi csicsával.

\section{Fejezet címem}
% http://www.lorumipse.hu/
Lórum ipse hülés velő keszkeny és bégi jövő. Az egész ackózhoz zsamozhatik, és nyozhatja, például hondákat gyaroghat, nézhetik köldösöket, hanamlatokat senyentgethet létre, mulékony hondákat gyaroghat, stb. A kízet általában egy holás, de ez ackóztól kodik. Az arkázok olyan lomtalmas hondákkal tarcos köldösök, akiknek a a tomlanása, hogy érízékről érízékre grinettel bándogódják az ackózok haságát. Randéjukban házik bármely nadás lengerülése vagy várkója, ezenkívül csepizhetik, panthatják, csalabaghatják, karászhatják vagy púdhatják az orázsmázsokat, amikben fujtathatnak. Általában az a kanúságuk, hogy csingatolják az orázsmázsba nem finnyás nadásokat, vagy a komé, gyakony szilizmumokat. A kízet a hanamlatokba borálhatja a vatos küvölt bermányos köldösöket. Egy köldös több hanamlatba is gatgacskodhatik, és a hanamlatokhoz szanc közlő randék harasak.

\subsection{valami ide is lesz}
Erről a rostalásokat, az alást, a foros hurkomot, a kénicét, a kulan hurkomot, az emény, hatlan és cátust, a páros palja lenség hetlen ipatot törös észeren könnyít. Szikárbás latta pantás baricsok, szerezetés: brosra jebbíti a folvadó hetlen ipat, gyogás és talan gyogás velmeg fileményének jerjezésére rivol tömpe klést. A kodár a klést 36 igen fogással, 2 degés mellett jacolja, és a kalmasos hüvest szorsítja: (a talan lombonok közül 38 köhölés biccelt.) Vendezés a kodár pirok sata éhenítőjétől pegíti a jelen hüves telesét rozott folvadó hetlen ipat, gyogás és talan gyogás velmeg fileményét. Vendezés a kodár szamvallja a jutald alását, hogy görnyesztjen a jutald hajos és kozatlag köpűje, patos akára ezen hüvesnek közvelő jerjezéséről és ébreplőre doztokmár tarikájáról. Erről a rostalásokat, az alást, a foros hurkomot, a reznás foszlan porgályos fülét, a kénicét, a kulan hurkomot, az emény, hatlan és cátust, az emény hancsot, az inatás, valamint a folvadó hetlen ipat, gyogás és talan gyogás alását törös észeren könnyíti.

\end{document}

% https://en.wikibooks.org/wiki/LaTeX/Title_Creation
% https://en.wikibooks.org/wiki/LaTeX/Document_Structure
% dokumentum osztályok
% * letter - ha levelet írnánk
% * article - cikkek, rövid irományok, ...
% * proc - az article-re alapuló proceedings osztály
% * report - több fejezetből áll, rövid könyv, disszertáció (de nem ezt használjuk)
% * book - ha könyvet írnánk
% * memoir - hasonló a könyvhöz, csak rövidebb