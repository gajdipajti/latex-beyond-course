\documentclass[12pt,a4paper]{article} % Állítsuk be a betűméretet és a papír méretet.
\usepackage[utf8]{inputenc}	% Manapság utf8 karakter kódolást használunk mindenhol.
\usepackage[T1]{fontenc}	% A magyar nyelv miatt szükséges a T1 betűkódolás

\title{A következő nagy dolog}
\author{Sztahanov Elemér\thanks{jómagam} \and Főnök Ferenc\thanks{Corresponding Author}}
\date{\today}

\begin{document}
\maketitle

Helló Világ! Ez csak egy mezítlábas \LaTeX{} dokumentum, semmi csicsával.

\textit{Megígérem hogy csak ezt fogom írni.}
\end{document}

% https://en.wikibooks.org/wiki/LaTeX/Title_Creation
% https://en.wikibooks.org/wiki/LaTeX/Document_Structure
% dokumentum osztályok
% * letter - ha levelet írnánk
% * article - cikkek, rövid irományok, ...
% * proc - az article-re alapuló proceedings osztály
% * report - több fejezetből áll, rövid könyv, disszertáció (de nem ezt használjuk)
% * book - ha könyvet írnánk
% * memoir - hasonló a könyvhöz, csak rövidebb